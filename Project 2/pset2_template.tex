\documentclass[12pt,letterpaper]{article}
\usepackage{fullpage}
\usepackage[top=2cm, bottom=4.5cm, left=2.5cm, right=2.5cm]{geometry}
\usepackage{amsmath,amsthm,amsfonts,amssymb,amscd}
\usepackage{lastpage}
\usepackage{enumerate}
\usepackage{fancyhdr}
\usepackage{mathrsfs}
\usepackage{xcolor}
\usepackage{graphicx}
\usepackage{listings}
\usepackage{hyperref}

\hypersetup{%
  colorlinks=true,
  linkcolor=blue,
  linkbordercolor={0 0 1}
}
 
\renewcommand\lstlistingname{Algorithm}
\renewcommand\lstlistlistingname{Algorithms}
\def\lstlistingautorefname{Alg.}

\lstdefinestyle{Python}{
    language        = Python,
    frame           = lines, 
    basicstyle      = \footnotesize,
    keywordstyle    = \color{blue},
    stringstyle     = \color{green},
    commentstyle    = \color{red}\ttfamily
}

\setlength{\parindent}{0.0in}
\setlength{\parskip}{0.05in}

% Edit these as appropriate
\newcommand\course{CSE 3500}
\newcommand\hwnumber{2}                  % <-- homework number
\newcommand\NetIDa{netid19823}           % <-- NetID of person #1

\pagestyle{fancyplain}
\headheight 35pt
\lhead{\NetIDa}
\chead{\textbf{\Large Homework \hwnumber}}
\rhead{\course \\ \today}
\lfoot{}
\cfoot{}
\rfoot{\small\thepage}
\headsep 1.5em

\begin{document}

\begin{center}
    \LARGE Problem Set
\end{center}


\section*{Problem 0 -- Recurrences (20\%)}

Give upper ($O(\cdot)$) asymptotic bounds for the following recurrences.
You may assume a $O(1)$ base case for small $n$.
Justify your answer by some combination of the following: deriving how much total work is done at an arbitrary level $k$, how many levels there are, and how much work is required to merge (function body). 
For each recurrence, state whether or not it is top-heavy, bottom-heavy, or even work.
Answers that only cite the Master theorem will not receive full credit.

\begin{enumerate}
    \item $T(n)=2T \left( \frac{n}{2} \right)+O(n)$
    \item $T(n)=2T \left( \frac{n}{2} \right)+O(1)$
    \item $T(n)=7T \left( \frac{n}{2} \right)+O(n^3)$
    \item $T(n)=7T \left( \frac{n}{2} \right)+O(n^2)$
    \item $T(n)=4T \left( \frac{n}{2} \right)+O(n^2 \sqrt{n})$
    \item $T(n)=4T \left( \frac{n}{2} \right)+O(n \log_2(n))$
\end{enumerate}

\newpage

\section*{Problem 1 -- Covering a chess board (20\%)}
You are given a $2^k \times 2^k$ board of squares (e.g. a chess board) with the top left square removed.
Prove, by giving a divide-and-conquer algorithm or argument, that you can exactly cover the entire board with L-shaped pieces (each covering 3 squares).

\newpage

\section*{Problem 2 -- Counting inverted pairs (20\%)}
You are given an unsorted list $L$ that has $k\geq0$ pairs of indices $i < j$ such that $L[i] > L[j]$. 
These are called \textit{inverted pairs}.
Develop an $O(n\log n)$ algorithm that counts the number of inverted pairs (i.e. compute the value $k$).

\newpage


\section*{Problem 3 -- Best subset problem (40\%)}
The \textit{best subset problem} is defined as, given a list $(x_1,x_2,\dots,x_n)$ of integers (which can be positive, negative, or zero), find $(i,j)$ such that $x_i+x_{i+1}+\dots+x_j$ is maximum for any $1\leq i \leq j \leq n$.
For example, if $n = 10$ and the input is $(4,-8,-5,8,-4,3,6,-3,2,-11)$ then the output is $x_4 + x_5 + x_6 + x_7 = 8-4 + 3 + 6 = 13$. 
\begin{enumerate}
    \item Develop an $O(n)$ algorithm for the related problem, \textit{best subset middle} or BSM.
    The input to BSM is a list $(x_1,x_2,\dots,x_n)$ of integers (which can be positive, negative, or zero) and the output is the maximum value of $x_i + x_{i+1} + \dots + x_j$ such that $[i,j]$ spans $\frac{n}{2}$, in other words, for all possibilities for $i$ and $j$ such that $1 \leq i \leq \frac{n}{2} \leq j \leq n$. 
\item Design a recursive algorithm for the best subset problem with runtime $O(n \log n)$ that uses the BSM function.
\item Argue that your algorithm is indeed correct and prove the runtime is $O(n \log n)$.
\item (Extra credit: 5pts) Design an algorithm for the best subset problem that has $O(n)$ runtime. Argue why your algorithm is correct and has $O(n)$ runtime.
\end{enumerate}


\end{document}
